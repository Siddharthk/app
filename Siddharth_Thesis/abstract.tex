Separation of images into components has been a widely used technique in the field of image processing and computer vision.
Many problems are solved by partitioning images into components. The simplest example
is the breaking down of image into Red, Green and Blue channels/components. The aim of this thesis is to use
the components of the image in better modeling of textures and text in scene images.
We first introduce a framework where separation of images 
into direct and global components helps in modeling of 3D textures.
These 3D textures are often described by parametric functions for each pixel, that
models the variation in its appearance with respect to varying lighting
direction. However, parametric models such as Polynomial Texture Maps (PTMs)
tend to smoothen the changes in appearance. Therefore we propose a technique to
effectively model natural material surfaces and their interactions with changing
light conditions. We show that the direct and global components of the image
have different nature, and when modeled separately, leads to a more accurate and
compact model of the 3D surface texture. Direct component is mainly affected by
structural properties of the surface and is therefore deals with phenomena like
shadows and specularity, which are sharply varying functions. The global
component is used to model overall luminance and color values, a smoothly
varying function. For a given lighting position, both components are computed
separately and combined to render a new image. This method models sharp shadows
and specularities, while preserving the structural relief and surface color.
Thus rendered image have enhanced photorealism as compared to images rendered by
existing single pixel models such as PTMs. 

We then show segmenation of natural scene text using the RGB components of the images.
Extracting text from scene images is a challenging task due to the variations in color, size, and font
of the text and the results are often affected by complex backgrounds, 
different lighting conditions, shadows and reflections.
A robust solution to this problem can significantly enhance
the accuracy of scene text recognition algorithms leading to a
variety of applications such as scene understanding, automatic
localization and navigation, and image retrieval. 
We propose a method to extract and binarize text from images
that contains complex background. We use Independent
Component Analysis (ICA) model to map out the text
region, which is inherently uniform in nature, while removing
shadows, specularity and reflections, which are included in
the background. The technique identifies the text regions from
the components extracted by ICA using a global thresholding
method to isolate the foreground text. We show the results
of our algorithm on some of the most complex word images
from the ICDAR 2003 Robust Word Recognition Dataset and
compare with previously reported methods.
